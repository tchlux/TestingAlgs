
\section{Abstract of Naive}
Each of high performance computing, cloud computing, and computer
security have their own interests in modeling and predicting the
performance of computers with respect to how they are configured. An
effective model might infer internal mechanics, minimize power
consumption, or maximize computational throughput of a given
system. This paper analyzes a four-dimensional dataset measuring the
input/output (I/O) characteristics of a cluster of identical computers
using the benchmark IOzone. The I/O performance characteristics are
modeled with respect to system configuration using multivariate
interpolation and approximation techniques. The analysis reveals that
accurate models of I/O characteristics for a computer system may be
created from a small fraction of possible configurations, and that
some modeling techniques will continue to perform well as the number
of system parameters being modeled increases. These results have
strong implications for future predictive analyses based on more
comprehensive sets of system parameters.


\section{Abstract of Scalable}
A rapid increase in the quantity of data available is allowing all fields of science to generate more accurate models of multivariate phenomena. Regression and interpolation become challenging when the dimension of data is large, especially while maintaining tractable computational complexity. This paper proposes three novel techniques for multivariate interpolation and regression that each have polynomial complexity with respect to number of instances (points) and number of attributes (dimension). Initial results suggest that these techniques are capable of effectively modeling multivariate phenomena while maintaining flexibility in different application domains.
