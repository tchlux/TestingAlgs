\documentclass{article}


%% Get access to mathematical symbols.
\usepackage{amssymb}
%% Get access to the \rightarrow command.
\usepackage{amsmath}
%% Use color links to make them clickable and easy to see.
\usepackage[colorlinks,allcolors=blue]{hyperref}
%% Set the font size of Section headers to be different.
\usepackage{sectsty}
\sectionfont{\fontsize{12}{15}\selectfont}
%% 

\begin{document}

\title{Algorithm for Constructing Piecewise Quintic \\ Monotone Interpolating Splines}
\author{Thomas C.H. Lux}
\maketitle

When provided data that has no assigned first and second derivative values, the derivative data is filled by a linear fit of neighboring data points. End points are set to be the slope between the end and its nearest neighbor.

The method finding a maximal value on a line is the Golden Section search. This will be referred to in pseudo code as \texttt{line\_search($a$,$b$)} where $a$ and $b$ are $n$-tuples for integer $n$.

After assigning function values and derivative values, an interpolating function is constructed by solving for the unique weights of a set of quintic B-splines in a linear system.

\vspace{10pt}

%% ----------------------------------------------------------------------

\section{Verifying Monotonicity of a Quintic Polynomial}
\label{is_monotone}
Let $f$ be a quintic polynomial over a closed interval $[x_0, x_1]$ $\subset \mathbb{R}$. Now $f$ is uniquely defined by the evaluation tuples $\big(x_0,$ $f(x_0),$ $f'(x_0),$ $f''(x_0)\big)$ and $\big(x_1,$ $f(x_1),$ $f'(x_1),$ $f''(x_1)\big).$ Assume without loss of generality that $f(x_0) < f(x_1),$ where the case of monotonic decreasing $f$ uses the negated the function values. This algorithm will determine whether or not $f$ is monotone increasing on the interval $[x_0, x_1].$

\vspace{10pt}
\hrule
\vspace{3pt}
\noindent\textbf{\textit{Algorithm 1a:}} \texttt{is\_monotone}
\vspace{3pt}
\hrule

\begin{itemize}
  \itemsep0pt
  \parskip0pt

\item[0:] \texttt{if ($f'(x_0) = 0$ or $f'(x_1) = 0$); return is\_monotone\_simplified}
\item[1:] \texttt{if ($f'(x_0) < 0$ or $f'(x_1) < 0$); return FALSE}
  \begin{itemize}
    \item[] \textit{This can be seen clearly from the fact that $f$ is analytic and there will exist some $0 < \epsilon < x_1 - x_0$ such that either $f'(x_0 + \epsilon)$ or $f'(x_1 - \epsilon)$ will be negative.}
  \end{itemize}

\item[2:] $A = f'(x_0)\frac{x_1 - x_0}{f(x_1) - f(x_0)}$
\item[3:] $B = f'(x_1) \frac{x_1 - x_0}{f(x_1) - f(x_0)}$
  \begin{itemize}
    \item[] \textit{The variables $A$ and $B$ correspond directly to the theoretical foundation for positive quartic polynomials laid in \cite{ulrich1994positivity}.}
  \end{itemize}
\item[8:] $\gamma_0 = 4 \frac{f'(x_0)}{f'(x_1)} (B/A)^{3/4}$
\item[9:] $\gamma_1 = \frac{x_1 - x_0}{f'(x_1)} (B/A)^{3/4}$
\item[4:] $\alpha_0 = 4 (B/A)^{1/4}$
\item[5:] $\alpha_1 = -\frac{x_1 - x_0}{f'(x_1)} (B/A)^{1/4}$
\item[6:] $\beta_0 = 30 - \frac{12 \big(f'(x_0) + f'(x_1)\big) (x_1 - x_0)}{\big(f(x_1) - f(x_0)\big) \sqrt{A}\sqrt{B}}$
\item[7:] $\beta_1 = \frac{-3 (x_1 - x_0)^2}{2 \big(f(x_1) - f(x_0)\big) \sqrt{A} \sqrt{B}} $
  \begin{itemize}
    \item[] \textit{The $\gamma,$ $\alpha,$ and $\beta$ terms with subscripts $0$ and $1$ are algebraic reductions of the original variables from \cite{ulrich1994positivity} that give the computation of each corresponding variable the form  $v = v_0 + v_1 c,$ where $c$ is a term involving only the second derivative values.}
  \end{itemize}
\item[11:] $\gamma = \gamma_0 + \gamma_1 f''(x_0)$
\item[10:] $\alpha = \alpha_0 + \alpha_1 f''(x_1)$
\item[12:] $\beta = \beta_0 + \beta_1 \big(f''(x_0) - f''(x_1)\big)$
\item[13:] \texttt{if $(\beta \leq 6)$; return $\alpha > - (\beta + 2) / 2$}
\item[14:] \texttt{else; return $\gamma > -2 \sqrt{\beta - 2}$ }

\end{itemize}
%% ----------------------------------------------------------------------


%% ----------------------------------------------------------------------
\subsection{Verifying Monotonicity of a Simplified Quintic}

Consider the same initial conditions outlined in Section \ref{is_monotone}.

\vspace{10pt}
\hrule
\vspace{3pt}
\noindent\textbf{\textit{Algorithm 1b:}} \texttt{is\_monotone\_simplified}
\vspace{3pt}
\hrule

\begin{itemize}
  \itemsep0pt
  \parskip0pt

\item Compute
\end{itemize}
%% ----------------------------------------------------------------------


%% ----------------------------------------------------------------------
\section{Enforcing Monotonicity of a Quintic Polynomial}


\vspace{10pt}
\hrule
\vspace{3pt}
\noindent\textbf{\textit{Algorithm 2a:}} \texttt{make\_monotone}
\vspace{3pt}
\hrule

\begin{itemize}
  \itemsep0pt
  \parskip0pt

\item Compute
\end{itemize}
%% ----------------------------------------------------------------------


%% ----------------------------------------------------------------------
\section{Enforcing Monotonicity of a Simplified Quintic}


\vspace{10pt}
\hrule
\vspace{3pt}
\noindent\textbf{\textit{Algorithm 2b:}} \texttt{make\_monotone\_simplified}
\vspace{3pt}
\hrule

\begin{itemize}
  \itemsep0pt
  \parskip0pt

\item Compute

  \begin{itemize}
    \item[] This follows the simplified conditions outlined in proposition 2 of \cite{schmidt1988positivity}.
  \end{itemize}

\end{itemize}
%% ----------------------------------------------------------------------


%% ----------------------------------------------------------------------
\section{Constructing a Piecewise Quintic Monotone Spline}
\label{monotone_spline}

Let $f: \mathbb{R} \rightarrow \mathbb{R}$ be a function in $\mathcal{C}^2.$ Proceed given evaluation tuples $\big(x_i,$ $f(x_i),$ $f'(x_i),$ $f''(x_i)\big)$ for $i = 0,\ldots,N$ such that $x_i < x_{i+1}$ and (without loss of generality) $f(x_i) \leq f(x_{i+1})$ for $i = 1,$ $\ldots,$ $N-1$. 

\vspace{10pt}
\hrule
\vspace{3pt}
\noindent\textbf{\textit{Algorithm 3:}} \texttt{monotone\_spline}
\vspace{3pt}
\hrule

\begin{itemize}
  \itemsep0pt
  \parskip0pt

\item Compute
\end{itemize}
%% ----------------------------------------------------------------------




\bibliographystyle{spmpsci}
\bibliography{pqmip}

\end{document}
