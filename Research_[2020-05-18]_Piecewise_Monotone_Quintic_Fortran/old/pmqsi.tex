\documentclass[12pt,letterpaper]{article}
\usepackage{fullpage}  % Make the margins smaller
\usepackage{parskip}   % Don't use indents in front of paragraphs
\usepackage{xcolor}    % Color package for coloring text.
\usepackage{fontenc}   % Use a font package.
\usepackage{libertine} % Pick the font.
\usepackage{hyperref} % Links 
\usepackage{amsfonts}  % Math font "mathbb"


%% Use a custom `algorithm` command for this paper.
%% This file defines a new command called `\algorithm` that has three
%% arguments. Usage is as follows:
%% 
%%    \algorithm{<title and arguments>}{<description>}{<body>}
%% 
%% If you want to label and reference, `\label{alg-name}` should be
%% put *inside* the `<body>` argument. By default the title and
%% arguments will be enclosed in a `\texttt`, override this behavior
%% with `\normalfont` inside of the `<title and arguments>` argument.
%% 
%% This is achieved by using the `\enumitem` package to define a
%% custom enumeration style. Inside the environment, `\item` or 
%% `\line` can be used to created a new numbered operation.
%% 
%% The commands `\indented{<text>}` and `\comment{<text>}` are 
%% provided for indenting and adding comments to the algorithm.
%% 


% Set up custom enumerate environments for algorithms.
\usepackage{enumitem} % <- custom enumerated environments.
%% Make a command for ':=' in math mode that has appropriate spacing.
\def\:{\mathrel{\mathord:\mathord=}}
%% Make a command for producing *comments* in the algorithms.
\newcommand{\comment}[1]{\par \vspace{3pt} {\normalfont \it #1} \vspace{3pt} \par}%
%% Make a command for producing *indented* sections in algorithms.
\newcommand{\indented}[1]{\begin{itemize} \item[] #1 \end{itemize}}%
%% Make a command for thick hrules
\newcommand{\thickrule}{\hrule%
  \hrule%
  \hrule%
}
%% Create a new counter for tracking algorithms.
\newcounter{algorithmCounter}
\newlist{algorithmList}{enumerate}{4} % {<name>}{<list style>}{<max depth}
%% Make the 'algorithmList' list:
%%   label   - have labels with a colon
%%   ref     - have references with only the number
%%   itemsep - have no extra separation between items
%%   parsep  - have no extra separation between paragraphs
%%   topsep  - have no extra space before the list begins
%%   font    - have a normal font for the counter
%%   before  - have a monospaced mode for item text by default 
%%             (use `\tt\fontseries{b}\selectfont` for bold)
\setlist[algorithmList]{label=\arabic*:,ref=\arabic*,itemsep=0pt,parsep=0pt,topsep=0pt,font=\normalfont,before=\tt}
%% Make a new block for containing an algorithm. Arguments are:
%% 
%%    \algorithm{title and arguments}{description}{body}
%% 
%% Any `\label{alg-name}` should be put *inside* the `body` argument.
%% 
\newcommand{\algorithm}[4]{
  \let\line\item
  %% Make some space before the algorithm
  \vspace{10pt}%
  \thickrule%
  \refstepcounter{algorithmCounter}%
  \textbf{Algorithm \thealgorithmCounter:} \texttt{#1} \\%
  #2 \par%
  \vspace{6pt}
  \hrule%
  \begin{algorithmList}%
    #3%
  \end{algorithmList}%
  \vspace{7pt}%
  \hrule%
  \vspace{10pt}%
}

\renewcommand*{\ttdefault}{courier}

\newcommand{\TODO}[1]{{\color{red} #1}}

\medskip

\title{Piecewise Monotone Quintic Spline Interpolants}
\author{Thomas C. H. Lux}
\date{}

\begin{document}
\thispagestyle{empty}
\begin{center}
    {\LARGE \textbf{Piecewise Monotone Quintic Spline Interpolants}} \hfill
    %% \href{https://tchlux.info}{\textbf{tchlux.info}} \hfill
    %% \textbf{804 317 8339} \hfill
    %% \href{thomas.ch.lux@gmail.com}{\textbf{thomas.ch.lux@gmail.com}}
    \\ \medskip\href{tchlux@vt.edu}{Thomas C.H. Lux}
    %% \rule{\textwidth}{1.5pt}
    \bigskip
\end{center}



%% \begin{tabular}{@{}l r@{}}
%%     \begin{tabular}{@{}l@{}}
%%       %% Hiring Committee \\
%%       Business \\
%%       City, State \\
%%       \today
%%     \end{tabular} &
%%     %% \hfill IMAGE
%% \end{tabular}

\medskip


\algorithm
    {is\_monotone$\big(x_0, x_1, f \big)$}
    {where $x_0$, $x_1 \in \mathbb{R}$, $x_0 < x_1$, and $f$ is an order six polynomial defined by $f(x_0)$, $Df(x_0)$, $D^2f(x_0)$, $f(x_1)$,
  $Df(x_1)$, $D^2f(x_1)$. Returns \texttt{TRUE} if $f$ is monotone on $[x_0,x_1]$.}
    {\label{alg:check-monotone}
  \item if $\big(f(x_0) = f(x_1)\big)$ and not $\big( 0 = Df(x_0) = Df(x_1) = D^2f(x_0) = D^2f(x_1) \big)$
    \indented{return FALSE}
    else if $\big(f(x_0) < f(x_1)\big)$ ; $s = 1$
    \\else ; $s = -1$
    \\end if
  \item if $\big(Df(x_0) < 0$ or $Df(x_1) < 0\big)$
    \indented{return FALSE}
    end if
    \comment{The necessity of these first two steps follows directly from the fact that $f$ is $C^2$. The next case is in accordance with a simplified condition for quintic monotonicity that reduces to one of cubic positivity studied in \cite{schmidt1988positivity}, where $\alpha$, $\beta$, $\gamma$, and $\delta$ are defined in terms of values and derivatives of $f$ at $x_0$ and $x_1$. Step \ref{alg1-alpha} checks for the necessary condition that $\alpha \ge 0$, Step \ref{alg1-beta} checks $\beta \geq \alpha$, and Step \ref{alg1-gamma} checks $\gamma \geq \delta$, all from \cite{schmidt1988positivity}. If all necessary conditions are met, then the order six piece is monotone and Step \ref{alg1-simplified-true} concludes this check.}
  \item if $\big(Df(x_0) = 0$ or $Df(x_1) = 0\big)$
  \item \indented{$w := x_0 - x_1$\\$v := f(x_0) - f(x_1)$}
  \item \indented{if $\big(D^2f(x_1) > {-4}Df(x_1) / w \big)$ return FALSE}
    \label{alg1-alpha}
  \item \indented{if $\big(D^2f(x_1) < (3w D^2f(x_0) - 24 Df(x_0) - 32
    Df(x_1) + 60v/w) / (5w) \big)$ return FALSE} \label{alg1-beta}
  \item \indented{if $\big(D^2f(x_0) < 3 Df(x_0) / w \big)$ return FALSE}
    \label{alg1-gamma}
  \item \indented{return TRUE} \label{alg1-simplified-true}
    end if
    \comment{The following code considers the remaining case where $Df(x_0) \neq 0$ and $Df(x_1) \neq 0$.}
  \item $\displaystyle A := Df(x_0)\frac{x_1 - x_0}{f(x_1) - f(x_0)}$
    \\ $\displaystyle B := Df(x_1) \frac{x_1 - x_0}{f(x_1) - f(x_0)}$
    \comment{The variables $A$ and $B$ correspond directly to the theoretical foundation for positive quartic polynomials established in \cite{ulrich1994positivity}, first defined after Equation (18).}
  \item $\displaystyle \gamma_0 := 4 \frac{Df(x_0)}{Df(x_1)} (B/A)^{3/4}$
    \\ $\displaystyle \gamma_1 := \frac{x_1 - x_0}{Df(x_1)} (B/A)^{3/4}$
    \\ $\alpha_0 := 4 (B/A)^{1/4}$
    \\ $\displaystyle \alpha_1 := -\frac{x_1 - x_0}{Df(x_1)} (B/A)^{1/4}$
    \\ $\displaystyle \beta_0 := 30 - \frac{12 \big(Df(x_0) + Df(x_1)\big) (x_1 - x_0)}{\big(f(x_1) - f(x_0)\big) \sqrt{A}\sqrt{B}}$
    \\ $\displaystyle \beta_1 := \frac{-3 (x_1 - x_0)^2}{2 \big(f(x_1) - f(x_0)\big) \sqrt{A} \sqrt{B}} $ \label{alg1-linearized-abg}
    \comment{The $\gamma$, $\alpha$, and $\beta$ terms with subscripts $0$ and $1$ are algebraic reductions of the simplified conditions for satisfying Theorem 2 in Equation (16) of \cite{ulrich1994positivity}. These terms with subscripts $0$ and $1$ make the computation of $\alpha$, $\beta$, and $\gamma$ functions of the second derivative terms, as seen in Step \ref{alg1-abg} below.}
  \item $\gamma := \gamma_0 + \gamma_1 D^2f(x_0)$
    \\ $\alpha := \alpha_0 + \alpha_1 D^2f(x_1)$
    \\ $\beta := \beta_0 + \beta_1 \big(D^2f(x_0) - D^2f(x_1)\big)$
    \label{alg1-abg}
  \item if $(\beta \leq 6)$ return $\big(\alpha > - (\beta + 2) / 2\big)$
    \\ else $\qquad$ return $\big(\gamma > -2 \sqrt{\beta - 2}\big)$
    \\ end if \label{alg1-last}
}


\medskip

\section*{Comments}

This is a final section discussing some of the finer points.

%% \medskip

%% With respect,\\
%% \bigskip\\
%% \bigskip\\
%% Thomas Lux
%% \vfill

\bibliographystyle{unsrt}
\bibliography{pmqsi}

\end{document}
