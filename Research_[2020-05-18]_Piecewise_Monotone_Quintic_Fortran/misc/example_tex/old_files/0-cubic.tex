\heading{1.1 \enspace Computing a Monotone Cubic Interpolant}

The current state-of-the-art monotone interpolating spline with a
mathematical software implementation is piecewise cubic, continuously
differentiable, and was first proposed in \cite{fritsch1980monotone}
then expanded upon in \cite{carlson1985monotone}. Let $\pi: x_0 = k_1
< k_2 < \cdots < k_n = x_1$ be a partition of the interval
$[x_0,x_1]$. Let $f: {\bbb R} \rightarrow {\bbb R}$ be $C^2$, and
$\bigl\{f(k_i)\bigr\}_{i=1}^n$ and $\bigl\{\Delta_i\bigr\}_{i=1}^n$ be
given sets of function and derivative values at the partition points
for a monotone function $f$. Either $f(k_i) \leq f(k_{i+1})$, $i=1$,
$\ldots$, $n-1$, and $\Delta_i\ge0$, $i=1$, $\ldots$, $n$, or $f(k_i)
\geq f(k_{i+1})$, $i=1$, $\ldots$, $n-1$, and $\Delta_i\le0$, $i=1$,
$\ldots$, $n$. Let $\hat f$ be a piecewise cubic polynomial defined in
each subinterval $I_i = [k_i, k_{i+1}]$ by
$$ h_i = k_{i+1} - k_{i}, \quad u(t) = 3t^2 - 2t^3, \quad p(t) = t^3 - t^2,$$
$$\hat f(x) = f(k_i) u\big((k_{i+1} - x) / h_i\big) + f(k_{i+1}) u\big((x - k_i) / h_i\big) $$
$$ - h_i\Delta_i p\big((k_{i+1}-x)/h_i\big) + h_i\Delta_{i+1} p\big((x-k_i)/h_i\big).$$ 
Notice that a trivially monotone spline results when $\Delta_i = 0$,
for $i = 1$, $\ldots$, $n$. However, such a spline has too many
{\it wiggles} for most applications. \cite{carlson1985monotone}
show that simple conditions on the derivative values can guarantee
monotonicity, and that these conditions can be enforced in a way that
ensures modifications on one interval will not break the monotonicity
of cubic polynomials over any neighboring intervals. If $f(k_i) =
f(k_{i+1})$, take $\Delta_i = \Delta_{i+1} =0$ and $\alpha =\beta =1$,
otherwise let $\alpha = \big(\Delta_i (k_{i+1}-k_i)\big) /
\big(f(k_{i+1}) - f(k_i)\big)$ and $\beta =
\big(\Delta_{i+1}(k_{i+1}-k_i)\big) / \big(f(k_{i+1}) -
f(k_i)\big)$. Monotonicity of a cubic polynomial over a subinterval
can be maintained by ensuring that $\alpha$ and $\beta$ reside in any
of the regions depicted in Figure \ref{fig:projection}.

\topinsert
\centerline{\epsfxsize=4.5truein \epsffile{vt_logo.eps}}
{\narrower\noindent\rmVIII Fig.\ 2.
These are the feasible regions of monotonicity for cubic splines and
the projections that make a cubic polynomial piece monotone. The
regions themselves are numbered 1--4 corresponding to their original
description in Fritsch and Carlson (1980). One point is projected onto
each region as a demonstration.
\par}
\endinsert

The actual region of monotonicity for a cubic polynomial is larger,
but projection of $(\alpha, \beta)$ into one of these regions ensures
that monotonicity will be achieved and not violated for neighboring
regions. The user must decide which region is most appropriate for the
projections based on the application, Fritsch and Carlson recommend
using Region 2.

While the cubic monotonicity case affords such a concise solution, the
region of monotonicity is not so simple in the quintic case. In the
next section, an algorithm for performing a projection similar to
those for cubic polynomials is proposed.
