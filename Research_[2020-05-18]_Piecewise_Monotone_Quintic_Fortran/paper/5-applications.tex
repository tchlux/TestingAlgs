\heading{5. PERFORMANCE AND APPLICATIONS}

This section contains graphs of sample {\tt MQSI} results given
various sources of data. Tables of computation times for various
problems sizes are also provided.

\midinsert
\centerline{\epsfxsize=4truein \epsffile{vis/2-piecewise-polynomial.eps}}
\centerline{\epsfxsize=4truein \epsffile{vis/3-large-tangent.eps}}
\centerline{\epsfxsize=4truein \epsffile{vis/4-signal-decay.eps}}
{\narrower\noindent\rmVIII Fig.\ 2.
{\ttVIII MQSI} results for three of the functions in the included test
suite. The {\itVIII piecewise polynomial} function (top) shows the
interpolant capturing local linear segments, local flats, and
alternating extreme points. The {\itVIII large tangent} (middle)
problem demonstrates outcomes on rapidly changing segments of data.
The {\itVIII signal decay} (bottom) alternates between extreme values
of steadily decreasing magnitude.
\par}
\endinsert

\midinsert
\centerline{\epsfxsize=4truein \epsffile{vis/5-random-monotone.eps}}
{\narrower\noindent\rmVIII Fig.\ 3.
The {\itVIII random monotone} test poses a particularly challenging
problem with large variations in slope. Notice that despite drastic
shifts in slope, the resulting {\ttVIII MQSI} provides smooth and
reasonable estimates to function values between data.
\par}
\endinsert
